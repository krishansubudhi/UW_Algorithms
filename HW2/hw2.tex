\documentclass{homeworg}

\title{CSEP 521, Winter 2021: Homework 2}
\author{Krishan Subudhi (ksubudhi@uw.edu) - 2040900}
\usepackage[table]{xcolor}
\usepackage{booktabs}
\usepackage{multirow}
\usepackage{pythonhighlight}
\usepackage{enumitem}

\begin{document}

\maketitle

\exercise
Office workers = $n$\\
Desks = $n$

Management assigns each worker to a random desk, without the requirement that only one person is assigned to a desk.

\begin{align*}
w_i &= \texttt{worker}_i \\
d_j &= \texttt{desk}_j \\
P(w_i, d_j) &= \frac{1}{n}
\end{align*}

\begin{enumerate}[label=\alph*)]
    \item \emph{What is the expected number of people getting their original desk back?}
    let $X_i$ be a random variable such that 
    \[
        X_i = \begin{cases}
			1, & \text{person $i$ gets his/her desk back}\\
            0, & \text{otherwise}
		 \end{cases}
    \]
    \begin{align*}
        & E(\text{number of people getting their original desk back}) \\
        & = \sum_{i=1}^n E(\text{person $i$ gets his/her desk back }) \\
        & = \sum_{i=1}^n E(X_i) \\
        & = n \ast E(X_i) \\
        & = n \ast (1 \ast P(w_i, d_i) + 0) \\
        & = n \ast \frac{1}{n}\\
        & = 1
    \end{align*}
    
    \textbf{Expected number of people getting their original desk back = 1}
    
    \item \emph{What is the probability of having zero people assigned to a particular desk?}
    
    for a particular desk $d_j$,
    \begin{align*}
        & P(\text{zero people assigned to}\ d_j) \\
        & = \prod_{i = 1}^{n} w_i \text{ not assigned to}\ d_j \\
        & = \prod_{i = 1}^{n} 1- P(w_i, d_j)\\
        & = \prod_{i = 1}^{n} 1-\frac{1}{n}\\
        & = \begin{pmatrix}
                \cfrac{n-1}{n}
            \end{pmatrix}^n
    \end{align*}
    
    \item \emph{What is the probability of having exactly one person assigned to a particular desk?}
    
    for a particular desk $d_j$,
    \begin{align*}
        & P(\text{exactly one person assigned}\ d_j) \\
        & = \sum_{i = 1}^{n} \text{only } w_i \text{ is assigned to}\ d_j \qquad  \text{... \textit{events for each i are mutually exclusive}} \\
        & = \sum_{i = 1}^{n} P(w_i, d_j) \prod_{k=1, k \ne j}^n (1- P(w_i, d_j))\\
        & = \sum_{i = 1}^{n} \frac{1}{n} \ast 
            \begin{pmatrix}
                \cfrac{n-1}{n}
            \end{pmatrix}^{n-1}\\
        & = n 
            \ast  \frac{1}{n} 
            \ast 
            \begin{pmatrix}
                \cfrac{n-1}{n}
            \end{pmatrix}^{n-1}\\
        & = \begin{pmatrix}
                \cfrac{n-1}{n}
            \end{pmatrix}^{n-1}
    \end{align*}
    
    \item What happens to the probabilities from parts b and c and n gets large?
    
    \textbf{part b}
    \begin{align*}
        & \lim_{n \to \infty} \left( \frac{n-1}{n} \right)^n \\
        =& \lim_{n \to \infty} \left( 1 -  \frac{1}{n} \right)^n \\
        =& \ \frac{1}{e} \label{eq}
    \end{align*}
    
    \textbf{part c}
    \begin{align*}
        & \lim_{n \to \infty} \left( \frac{n-1}{n} \right)^{n-1} \\
        =& \lim_{n \to \infty} \left( 1 -  \frac{1}{n} \right)^{n-1} \\
        =& \lim_{n \to \infty} \left( 1 -  \frac{1}{n} \right)^{n} 
        \ast \lim_{n \to \infty} \left( 1 -  \frac{1}{n} \right)^{-1} \\\\
        =& \ \frac{1}{e} \ast 1 \\
        =&\ \frac{1}{e}
    \end{align*}
\end{enumerate}
\newpage

\exercise
\emph{Show that if the pivot is always the median, then the selection algorithm makes (about) 2n comparisons. 
find the maximum element.}

For this problem, I am selecting the median during each pivot selection step. Using that median as pivot, quick select is applied to find the max element.

\begin{table}[htbp]
\label{q2:result}
\caption{Results of quick select to find max}

    \begin{tabular}{llccc}
    \toprule
            &                                       &  exp partitions & exp comparisons & exp runtime \\
    n & algorithm &                      &                      &                  \\
    \midrule
    \multirow{4}{*}{1000} & Quick select random pivot &                  7.2 &               2.17 n &   0.0029 s \\
            & Quick select middle index &                 10.0 &               \cellcolor{green} 1.98 n &   0.0060 s \\
            & Quick select Median from sorting &                 10.0 &               \cellcolor{green} 1.98 n &   0.0100 s \\
            & Quick select Median from quick select &                  9.0 &               \cellcolor{green} 1.98 n &   0.0125 s \\
    \cline{1-5}
    \multirow{4}{*}{10000} & Quick select random pivot &                  9.9 &               2.15 n &   0.0165 s \\
            & Quick select middle index &                 14.0 &               \cellcolor{green} 2.00 n &   0.0158 s \\
            & Quick select Median from sorting &                 14.0 &               \cellcolor{green} 2.00 n &   0.0184 s \\
            & Quick select Median from quick select &                 13.0 &               \cellcolor{green} 2.00 n &   0.0586 s \\
    \cline{1-5}
    \multirow{4}{*}{100000} & Quick select random pivot &                 11.9 &               2.63 n &   0.2300 s \\
            & Quick select middle index &                 17.0 &               \cellcolor{green} 2.00 n &   0.1939 s \\
            & Quick select Median from sorting &                 17.0 &               \cellcolor{green} 2.00 n &   0.2748 s \\
            & Quick select Median from quick select &                 16.0 &               \cellcolor{green} 2.00 n &   0.8080 s \\
    \cline{1-5}
    \multirow{4}{*}{1000000} & Quick select random pivot &                 13.7 &               1.83 n &   1.7610 s \\
            & Quick select middle index &                 20.0 &               \cellcolor{green} 2.00 n &   0.9028 s \\
            & Quick select Median from sorting &                 20.0 &               \cellcolor{green} 2.00 n &   1.1352 s \\
            & Quick select Median from quick select &                 19.0 &               \cellcolor{green} 2.00 n &   3.8964 s \\
    \bottomrule
    \end{tabular}

\end{table}

if the pivot is always the median (last 3 rows for each n), then the selection algorithm makes (about) $2n$ comparisons. If the pivot is selected at random (first row for each n) the number of comparisons are more than $2n$

For selecting the median 3 algorithms were used. 
\begin{enumerate}
    \item select the middle index (only works for identity permutation)
    \item Python inbuilt sort (mix of merge and selection sort)
    \item quick select with random pivot
\end{enumerate}

Note:
The median selection algorithm was not the focus of the assignment. Hence comparisons done to find the median are not included in the result. But the run time includes time taken to select median at each stage. python inbuilt quick sort is faster because it's implemented in C. So even though it's $n (log n)$ run time at each stage, it's faster than my quick select algorithm which is written in pure python.

The algorithm uses identity permutation as the data but is tested for non identity permutations as well and producing ma
\begin{python}

import random
import pandas as pd
import collections
class QuickSelect:
    def __init__(self):
        self.comparison_count = 0
        self.partition_count = 0

    def kth_largest(self, data :list, k: int):
        '''
        Find k th largest element in data. K should be between 1 and length of data (inclusive)
        '''
        self.data = data
        self.comparison_count = 0
        self.partition_count = 0
        assert k<=len(self.data) and k>0, 'K out of range'
        element =  self._kth_largest(k, 0, len(data)-1)
        return element
    
    def _kth_largest(self, k: int, start : int, end : int):
        """
        Given a list, find it's kth largest element using Quick select algorithm
        """
        # 1. select a random pivot
        pivotIndex = self.get_pivot_index(start, end)
        # print('Before partition start, end, pivotIndex ',  start, end, pivotIndex)
        pivotIndex = self.partition(pivotIndex, start, end)
        # print(f'After partition , data = {data}, pivotIndex = {pivotIndex}')

        if end-pivotIndex >=k:
            return self._kth_largest(k, pivotIndex+1, end)
            
        elif end-pivotIndex+1 == k:
            return pivotIndex, self.data[pivotIndex]
        else:
            return self._kth_largest(k-1-(end-pivotIndex), start , pivotIndex-1)

    def partition(self, pivotIndex, start, end):
        self.partition_count += 1
        pivotValue = self.data[pivotIndex]
        # Move pivot to right
        self.swap(end , pivotIndex)
        pivotIndex = end
        leftIndex = start
        # Loop till end-1.
        for i in range(start, end):
            self.comparison_count += 1
            if data[i] < pivotValue:
                self.swap(leftIndex, i)
                leftIndex += 1
        self.swap(leftIndex, pivotIndex)
        return leftIndex
    
    def swap(self, i , j):
        t = self.data[j]
        self.data[j] = self.data[i]
        self.data[i] = t

    def get_pivot_index(self,start, end):
        randomIndex = random.randint(start,end)
        # print('pivot index, value =',randomIndex,self.data[randomIndex])
        return randomIndex

    def __str__(self):
        return 'Quick select random pivot'

class QuickSelectMedianSort(QuickSelect):
    def get_pivot_index(self, start, end):
        randomIndex = range(start, end+1)
        # Use inbuilt sort
        randomIndex = sorted(randomIndex, key = lambda index: self.data[index])
        median = randomIndex[(end-start)//2]
        # print(f'{start}:{end} - pivot index, value =',median,self.data[median])
        return median

    def __str__(self):
        return f'Quick select Median from sorting'

class QuickSelectMedianQS(QuickSelect):
    def get_pivot_index(self, start, end):
        q= QuickSelect()
        q.data = self.data
        median, medianval = q._kth_largest( (end-start)//2+1, start, end) # this manipulates the input though
        # print(f'{start}:{end} - comparisons = {q.comparison_count} , pivot index, value =',median,self.data[median])
        return median

    def __str__(self):
        return f'Quick select Median from quick select'

class QuickSelectMiddleIndex(QuickSelect):
    # Only works with identity permutation
    def get_pivot_index(self, start, end):
        median = start + (end-start)//2
        # print(f'{start}:{end} - pivot index, value =',median,self.data[median])
        return median

    def __str__(self):
        return f'Quick select middle index'

import time
if __name__ == '__main__':
    import sys
    algos = [QuickSelect(), QuickSelectMiddleIndex(), QuickSelectMedianSort(), QuickSelectMedianQS()]
    iterations  =10
    results = []
    for n in [1_000, 10_000, 100_000,1_000_000]:
        data = [i for i in range(1,n+1)]
        for q in algos:
            print(q)
            random.seed(48)
            comparisons = 0
            partitions = 0
            start = time.time()
            for _ in range(iterations):
                # data = [random.randint(0,n) for i in range(1,n+1)]
                k = 1 #max
                element = q.kth_largest(data, k)
                print(f'q: max index =  {element[0]}, max value = {element[1]}')
                comparisons += q.comparison_count
                partitions += q.partition_count
            end = time.time()
            results.append(pd.Series({
                'n':n,
                'algorithm':str(q),
                'expected partitions':partitions/iterations,
                'expected comparisons' :f'{comparisons/iterations/n:.2f} n',
                'expected runtime' : f'{(end-start)/iterations:.4f} s'
            }))
    df = pd.DataFrame(results).set_index(['n','algorithm'])
    print(df)
    print(df.to_latex(multirow = True, multicolumn_format = 'c'))

\end{python}
\end{document}