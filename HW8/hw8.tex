\documentclass{homeworg}

\title{CSEP 521, Winter 2021: Homework 8}
\author{Krishan Subudhi (ksubudhi@uw.edu) - 2040900}
\usepackage[table]{xcolor}
\usepackage{booktabs}
\usepackage{multirow}
\usepackage{pythonhighlight}
\usepackage{enumitem}
\usepackage{array}

\usepackage{graphicx}
\graphicspath{ {./images/} }
\usepackage{placeins}

\let\Oldsubsection\subsection
\renewcommand{\subsection}{\FloatBarrier\Oldsubsection}
\newcommand\numberthis{\addtocounter{equation}{1}\tag{\theequation}}

\begin{document}

\maketitle

\exercise
\emph{Q: Is the $L^{1/2}$ norm a proper distance function (a metric)? Prove or disprove.}

\begin{equation}
\label{eq:1}
    ||(x, y)||_{1/2} =  (\sqrt{|x|}+\sqrt{|y|})^2
\end{equation}

The properties of a proper distance function are:
\begin{align*}
1. d(x,y) &= 0 iff x = y\\
2. d(x,y) &= d(y,x) \\
3. d(x,y) &\le d(x,z) + d(z,y)\\ 
4. d(x,y) &\ge 0
\end{align*}
$L^{1/2}$  does not always satisfy property 3 for all combination of $x$ and $y$. Hence it is not a proper distance function.

Let's take one example in 2d.
\begin{align*}
x = (0,1)\\
y = (1,0)\\
z = (0,0)\\
d(x,y) = (1+1)^2 = 4\\
d(x,z) = (0+1)^2 = 1\\
d(y,z) = (1+0)^2 = 1\\
\\
d(x,z) +d(y,z)  = 2\\
d(x,y) > d(x,z) + d(z,y)
\end{align*}
    
Since for the above example, the property $d(x,y) \le d(x,z) + d(z,y)$ does not hold true, $L^{1/2}$ norm is not a proper distance function.

\newpage
\exercise

\emph{Q. Suppose that $|U| = n$ and you select random subsets, $A \subseteq U$ and $B \subseteq U$ with $|A| = m$ and $|B| = m$. What is the expected size of $A \cap B$?}

Let $X_i$ be an indicator random variable with values
$$
X_i=\begin{cases}
			1, & \text{if $i^{th}$ element is present in both A and B}\\
            0, & \text{otherwise}
		 \end{cases}
$$
\begin{align*}
E[|A \cap B|] &= E[\sum_{i=1}^n{X_i}]\\
&=\sum_{i=1}^n{E[X_i]}\\
&= n \ast E[X_i]\\
&= n \ast P(X_i =1)\\
&= n \ast P(i \ in \ A) \ast P(i \ in \ B)\\
&= n \ast m/n \ast m/n\\
&= \frac{m^2}{n}
\end{align*}

\emph{Q: Give an expression for the value of the Jaccard similarity of A and B if $m = n/k$}
$$
J(A,B) = \frac{|A \cap B|}{|A \cup B|}
$$
if $m = n/k$, 
\begin{align*}
    E[|A \cap B|] &=  \frac{m^2}{n} = \frac{n}{k^2}\\
    E[|A \cup B|] &=  E[|A| + |B| - |A \cap B|]\\
    & = \frac{2n}{k} - \frac{n}{k^2}\\
    & = \frac{n}{k^2} (2k- 1)\\
    J(A,B) &= \frac{|A \cap B|}{|A \cup B|}\\
    &=\frac{1}{2k-1}
\end{align*}

Hence $J \alpha \frac{1}{k}$. This means that for uncorrelated data, as $k$ decreases (i.e. subset size increases), Jaccard similarity increases.
\end{document}