\documentclass{homeworg}

\title{CSEP 521, Winter 2021: Homework 3}
\author{Krishan Subudhi (ksubudhi@uw.edu) - 2040900}
\usepackage[table]{xcolor}
\usepackage{booktabs}
\usepackage{multirow}
\usepackage{pythonhighlight}
\usepackage{enumitem}
\usepackage{array}

\usepackage{graphicx}
\graphicspath{ {./images/} }
\usepackage{placeins}

\let\Oldsubsection\subsection
\renewcommand{\subsection}{\FloatBarrier\Oldsubsection}

\begin{document}

\maketitle

\exercise
Unique keys possible with 128 bits = d
\begin{align*}
d &=  2^{128}\\
p &= 1\% =0.01\\
n(p;d) &= \sqrt{2d \ln{\frac{1}{1-p}}}\\
& = \sqrt{2 \ast 2^{128} \ln{\frac{1}{1-0.01}}}
\end{align*}
Here $n$ is the number of keys that need to be generated until there is a $1\%$ probability that two keys are same. 

$n$ is the number of transactions. 1 billion users are generating 1 thousand transactions per day. Hence the user can record data until $n/1E9/1000/365$ years.

\begin{python}
n/1E9/1000/365 = math.sqrt(math.pow (2,129) * math.log(1/0.99))/1E12/365
\end{python}
 = $7165.263$ years
 
Hence the user can record data until 7165 years until there is 1\% chance of having two transactions with the same key.
\newpage

\exercise
\begin{enumerate}[label=\alph*)]
\item 
$m$ is ranked first on $w$'s preference list.\\
$w$ is ranked first on $m$'s preference list.

We have to prove that for every stable matching for $I(M,W)$, $(m,w)$ must be the matching.

Let's prove this through contradiction. Let's assume that in one of the stable matching instances $S$, $m$ is matched with $w'$  and $w$ is matched with $m'$.

However, since $m$ prefers $w$ over $w'$ and $w$ prefers $m$ over $m'$, $(m,w')$ , $(m',w)$ is an instability.

This contradicts our claim that $S$ is stable and hence proves that every stable matching instance must match $m$ with $w$.

\item In stable matching problem $I =(M,W)$ ,\\
Preference list of all $m \in M$ is $[w_1,w_2,. .., w_n]$


For a particular $I$, preference list is fixed for for every $m \in M$ and every woman $w \in W$ 

We have to prove that \emph{There is a unique solution to this instance}

Let's say in the stable matching problem $I$, $w_1$ prefers $m_i$ over $m_j$

Let assume that there are two stable matching possible for instance $I$ - $S$ and $S'$.  In $S$, $w_1$ is matched with $m_i$ and in $S'$ $w_1$ is matched with $m_j$.

For $w', w'' \in [w_2,w_3,. .., w_n]$
\begin{align*}
    S : (m_i,w_1),(m_j,w')\\
    S' : (m_j,w_1),(m_i,w'')\\
\end{align*}

Since $w_1$ prefers $m_i$ over over $m_j$, and $m_i$  prefers $w_1$ over $w'$ and $w''$, $S'$ will not be stable match - contradicting our initial assumption. Hence the only match possible for $w_1$ is $(w_1,m_i)$.

Now since $(w_1,m_i)$ is the only stable match possible for $w_1$ and $m_i$, we can remove $w_1$ and $m_i$ from the preference lists and match others. After removing  $w_1$, the preference list for men becomes $[w_2,. .., w_n]$. Applying the same logic as above, we can prove that there is an unique stable match for $w_2$ too. hence it can be proven by induction that each $w_i$ will have an a unique stable match. Since each $w_i$ has a unique stable match, there is only one unique stable solution possible for $I$.

\end{enumerate}

\newpage

\exercise
\emph{Q: Show that a participant can improve its outcome by lying about its preferences.}

In Gale-Shapley algorithm, women can imrpove their chances of achieving a better match if they switch the order of their preferences in some cases. Let's say the preference for women is in the following order:

\begin{align*}
    w &: [m'' , m , m']\\
    w'&: [m, m'', m']\\
    w''&:[m, m'', m']\\
\end{align*}
Let's say preference list of men is
\begin{align*}
    m &: [w , w' , w'']\\
    m'&:[w , w' , w'']\\
    m''&: [w', w, w'']
\end{align*}

If we run the Gale-Shapley on the preference lists above, we obtain the final match to be $m-w, m''-w', m'-w''$

In this case $m$ is higher in preference list of other women. But since $w$ is first preference of $m$, and $m$ proposes first in Gayle Shapely algorithm, $w$  will accept $m$'s proposal and $w'$ will accept $m''$'s proposal. The algorithm still produces stable match even though $w$ does not get her first preference. 

Now let's say $w$ lies about her preference. 

\begin{align*}
    w &: [m'' , m' , m]\\
    w'&: [m, m'', m']\\
    w''&:[m, m'', m']\\
\end{align*}

If we run the Gale-Shapley on the updated preference list of women if $w$ lies, we obtain the match to be $m-w', m''-w, m'-w''$, noticing that w1 ends up with her first preference.

Hence $w$ was matched $m''$ which she prefers higher than both $m$ and $m'$. This proves that lying about preference can increase the chances to get a better match. 
\newpage

\exercise
\emph{Q) Implement the stable matching algorithm.}

Here I have implemented the Gayle Shapely algorithm. Men propose to women and women accept the proposal if they are unmatched or the proposal is preferred compared to the current match. The specific implementation uses a set of unmatched men.
\begin{verbatim}

men
[[2 1 3 0] 
 [0 1 3 2] 
 [0 1 2 3] 
 [0 1 2 3]]

women      
[[0 2 1 3] 
 [2 0 3 1] 
 [3 2 1 0] 
 [2 3 1 0]]
0 proposes to 2 [2,-1] Accepted
1 proposes to 0 [0,-1] Accepted
2 proposes to 0 [0,1] Accepted
3 proposes to 0 [0,2] Rejected
3 proposes to 1 [1,-1] Accepted
1 proposes to 1 [1,3] Rejected
1 proposes to 3 [3,-1] Accepted

matches
 [2, 3, 0, 1]

\end{verbatim}
\begin{python}

# question4.py

import numpy as np
import dataclasses


@dataclasses.dataclass
class Person:
    index: int
    pref_list: list
    current_match: int = -1  # accepted

class Man(Person):
    total_proposed: int = 0  # may or may not be accepted
    def get_next_woman(self):
        return self.pref_list[self.total_proposed]
    @property
    def mrank(self):
        return self.total_proposed


class Woman(Person):
    def __init__(self, *args, **kwargs):
        super().__init__(*args, **kwargs)
        self.pref_dict = {m: i for i, m in enumerate(self.pref_list)}
        del self.pref_list
    def __str__(self):
        return super().__str__() + str(self.pref_dict)
    def __repr__(self):
        return super().__repr__() + str(self.pref_dict)
    @property
    def wrank(self):
        return self.pref_dict[self.current_match]


class GayleShapely:
    def __init__(self, m: list, w: list, trace=True):

        m = np.array(m)
        w = np.array(w)
        self.men = [Man(i, pref) for i, pref in enumerate(m)]
        self.women = [Woman(i, pref) for i, pref in enumerate(w)]
        self.trace = trace
        
        self.free_men = set([m.index for m in self.men])
        if trace:
            print('men')
            print(m)
            print('women')
            print(w)

    def _get_next_free_m(self):
        return self.men[self.free_men.pop()] if len(self.free_men) >0 else None

    def _matched(self, man: Man, woman: Woman):
        man.current_match = woman.index
        if woman.current_match > -1:
            self.men[woman.current_match].current_match = -1
            self.free_men.add(woman.current_match)
        woman.current_match = man.index

        


    def _propose(self, man, woman):
        man.total_proposed += 1
        result = "Rejected"
        woman_prev_match = woman.current_match
        if (
            woman.current_match == -1
            or woman.pref_dict[man.index] < woman.pref_dict[woman.current_match] #this is rank so lower is better
        ):
            self._matched(man, woman)
            result = "Accepted"

        if self.trace:
            print(
                f"{man.index} proposes to {woman.index} [{woman.index},{woman_prev_match}] {result}"
            )
        return result == "Accepted"

    def match(self):
        """
        m and w are objects
        """
        m = self._get_next_free_m()
        
        while m is not None:
            accepted = False
            while not accepted:
                # select next woman
                w = self.women[ m.get_next_woman()]
                # _propose
                accepted = self._propose(m, w)
            m = self._get_next_free_m()
        if self.trace:
            print('matches\n',self.get_matches())

    def get_matches(self):
        return [m.current_match for m in self.men]

    def get_mranks(self):
        return [m.mrank for m in self.men]
    
    
    def get_wranks(self):
        return [w.wrank for w in self.women]

    @property
    def MGoodness(self):
        return sum(self.get_mranks())/len(self.men)
    
    @property
    def WGoodness(self):
        return sum(self.get_wranks())/len(self.women)

def main():
    m = [
        [2,1,3,0],
        [0,1,3,2],
        [0,1,2,3],
        [0,1,2,3]
    ]

    w = [
        [0,2,1,3],
        [2,0,3,1],
        [3,2,1,0],
        [2,3,1,0]
    ]
    algo = GayleShapely(m,w)
    algo.match()

if __name__ == "__main__":
    main()
\end{python}

\newpage

\exercise

\emph{Q. Write an input generator which creates completely random preference lists, so that each M has a random permutation of the W’s for preference, and vice-versa}
\begin{python}
from question4 import Man, Woman, GayleShapely
import random
import numpy as np
import time
import pandas as pd
import matplotlib.pyplot as plt
import math
plt.style.use('ggplot')

def generate_random_permutations(n):
    return [np.random.permutation(range(n)).tolist() for i in range(n)]

def main():
    results = {}
    for n in [8000]:
        random.seed(46)
        
        m = generate_random_permutations(n)
        w = generate_random_permutations(n)

        algo = GayleShapely(m, w, trace = False)
        
        start = time.time_ns()/1000
        algo.match()
        end = time.time_ns()/1000
        
        total_time = end-start
        
        results[n]= pd.Series({
                'time_micros':total_time,
                'MGoodness':algo.MGoodness,
                'WGoodness':algo.WGoodness,
                'time_ms/CouponCollector':total_time/theoritical_cn(n)
            })
    df =  pd.DataFrame(results).T
    return df

def theoritical_cn(n):
    return n * math.log(n) + 0.57 * n

if __name__=='__main__':
    result = main()
    print(result)

    fig, axes = plt.subplots(2,2, sharex = True)
    axes = axes.reshape(-1)
    # print(axes)
    for i, col in enumerate(result.columns):
        axes[i].plot(result.index, result[col])
        axes[i].set_ylabel(col)
        axes[i].set_xlabel('n')
    plt.show()
\end{python}

\begin{figure}[h]
    \centering
    \includegraphics[width=1\textwidth]{images/q5.png}
    \caption{Growth rate of Goodness and run time}
    \label{fig:q5}
\end{figure}

\begin{table}
\centering
\caption{Question 5 results}
\label{table:q5}
\small\addtolength{\tabcolsep}{-1pt}
\begin{tabular}{lrrrrrrr}
\toprule
{} &  time($\mu$s) &  MGoodness &  WGoodness &  time/CouponColl &  MGoodness/logn &  Wgoodness/n &  time/nlogn \\
n    &              &            &            &                              &                 &              &             \\
\midrule
500  &      6000.25 &   7.936000 &    64.3340 &                     1.768783 &        1.276991 &     0.128668 &    1.931015 \\
1000 &     17031.25 &   9.421000 &   105.5240 &                     2.277589 &        1.363829 &     0.105524 &    2.465526 \\
2000 &     44999.00 &   8.495500 &   239.4845 &                     2.753613 &        1.117696 &     0.119742 &    2.960109 \\
3000 &     58276.50 &   8.398667 &   362.2550 &                     2.265003 &        1.048998 &     0.120752 &    2.426256 \\
5000 &    181921.75 &   9.585400 &   519.9838 &                     4.003915 &        1.125418 &     0.103997 &    4.271871 \\
8000 &    270089.50 &   7.641625 &  1026.2355 &                     3.532541 &        0.850279 &     0.128279 &    3.756587 \\
\bottomrule
\end{tabular}
\end{table}

\begin{enumerate}
    \item Q. how does the goodness change for M and W?

MGoodness grows at the rate of $O(logn)$ while WGoodness grows at a rate less than $O(n)$. The growth rates might not be exact but the constant ratio gives an indication of the upper bound. Clearly the algorithm is not favourable for women.

    \item Q. How does the run time of the code vary as a function of n?\\

Run time grows approximately at a rate of $O(n log n )$

    \item Q. How do your results relate to result from the coupon\\ collector problem?

The ratio of runtime and Coupon Collector theoritical value is almost a constant at every n. Approximately the growth rates seem to match. 

Note: the constant values are slightly higher for higher $n$. But there can be other constraints like ineffecient memory management, processor throttling which might increase runtime at higher $n$. But the overall constant ratio makes this algorithm runtime comparable with coupon collector. 
\end{enumerate}
\end{document}