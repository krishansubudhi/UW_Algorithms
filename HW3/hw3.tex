\documentclass{homeworg}

\title{CSEP 521, Winter 2021: Homework 3}
\author{Krishan Subudhi (ksubudhi@uw.edu) - 2040900}
\usepackage[table]{xcolor}
\usepackage{booktabs}
\usepackage{multirow}
\usepackage{pythonhighlight}
\usepackage{enumitem}

\usepackage{graphicx}
\graphicspath{ {./images/} }
\usepackage{placeins}
\let\Oldsubsection\subsection
\renewcommand{\subsection}{\FloatBarrier\Oldsubsection}

\begin{document}

\maketitle

\exercise
Unique keys possible with 128 bits = d
\begin{align*}
d &=  2^{128}\\
p &= 1\% =0.01\\
n(p;d) &= \sqrt{2d \ln{\frac{1}{1-p}}}\\
& = \sqrt{2 \ast 2^{128} \ln{\frac{1}{1-0.01}}}
\end{align*}
Here $n$ is the number of keys that need to be generated until there is a $1\%$ probability that two keys are same. 

$n$ is the number of transactions. 1 billion users are generating 1 thousand transactions per day. Hence the user can record data until $n/1E9/1000/365$ years.

\begin{python}
n/1E9/1000/365 = math.sqrt(math.pow (2,129) * math.log(1/0.99))/1E12/365
\end{python}
 = $7165.263$ years
 
Hence the user can record data until 7165 years until there is 1\% chance of having two transactions with the same key.
\newpage

\exercise
\end{document}