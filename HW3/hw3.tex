\documentclass{homeworg}

\title{CSEP 521, Winter 2021: Homework 3}
\author{Krishan Subudhi (ksubudhi@uw.edu) - 2040900}
\usepackage[table]{xcolor}
\usepackage{booktabs}
\usepackage{multirow}
\usepackage{pythonhighlight}
\usepackage{enumitem}

\usepackage{graphicx}
\graphicspath{ {./images/} }
\usepackage{placeins}
\let\Oldsubsection\subsection
\renewcommand{\subsection}{\FloatBarrier\Oldsubsection}

\begin{document}

\maketitle

\exercise
Unique keys possible with 128 bits = d
\begin{align*}
d &=  2^{128}\\
p &= 1\% =0.01\\
n(p;d) &= \sqrt{2d \ln{\frac{1}{1-p}}}\\
& = \sqrt{2 \ast 2^{128} \ln{\frac{1}{1-0.01}}}
\end{align*}
Here $n$ is the number of keys that need to be generated until there is a $1\%$ probability that two keys are same. 

$n$ is the number of transactions. 1 billion users are generating 1 thousand transactions per day. Hence the user can record data until $n/1E9/1000/365$ years.

\begin{python}
n/1E9/1000/365 = math.sqrt(math.pow (2,129) * math.log(1/0.99))/1E12/365
\end{python}
 = $7165.263$ years
 
Hence the user can record data until 7165 years until there is 1\% chance of having two transactions with the same key.
\newpage

\exercise
\begin{enumerate}[label=\alph*)]
\item 
$m$ is ranked first on $w$'s preference list.\\
$w$ is ranked first on $m$'s preference list.

We have to prove that for every stable matching for $I(M,W)$, $(m,w)$ must be the matching.

Let's prove this through contradiction. Let's assume that in one of the stable matching instances $S$, $m$ is matched with $w'$  and $w$ is matched with $m'$.

However, since $m$ prefers $w$ over $w'$ and $w$ prefers $m$ over $m'$, $(m,w')$ , $(m',w)$ is an instability.

This contradicts our claim that $S$ is stable and hence proves that every stable matching instance must match $m$ with $w$.

\item In stable matching problem $I =(M,W)$ ,\\
Preference list of all $m \in M$ is $[w_1,w_2,. .., w_n]$


For a particular $I$, preference list is fixed for for every $m \in M$ and every woman $w \in W$ 

We have to prove that \emph{There is a unique solution to this instance}

Let's say in the stable matching problem $I$, $w_1$ prefers $m_i$ over $m_j$

Let assume that there are two stable matching possible for instance $I$ - $S$ and $S'$.  In $S$, $w_1$ is matched with $m_i$ and in $S'$ $w_1$ is matched with $m_j$.

For $w', w'' \in [w_2,w_3,. .., w_n]$
\begin{align*}
    S : (m_i,w_1),(m_j,w')\\
    S' : (m_j,w_1),(m_i,w'')\\
\end{align*}

Since $w_1$ prefers $m_i$ over over $m_j$, and $m_i$  prefers $w_1$ over $w'$ and $w''$, $S'$ will not be stable match - contradicting our initial assumption. Hence the only match possible for $w_1$ is $(w_1,m_i)$.

Now since $(w_1,m_i)$ is the only stable match possible for $w_1$ and $m_i$, we can remove $w_1$ and $m_i$ from the preference lists and match others. After removing  $w_1$, the preference list for men becomes $[w_2,. .., w_n]$. Applying the same logic as above, we can prove that there is an unique stable match for $w_2$ too. hence it can be proven by induction that each $w_i$ will have an a unique stable match. Since each $w_i$ has a unique stable match, there is only one unique stable solution possible for $I$.

\end{enumerate}

\newpage

\exercise
\emph{Q: Show that a participant can improve its outcome by lying about its preferences.}

In Gale-Shapley algorithm, women can imrpove their chances of achieving a better match if they switch the order of their preferences in some cases. Let's say the preference for women is in the following order:

\begin{align*}
    w &: [m'' , m , m']\\
    w'&: [m, m'', m']\\
    w''&:[m, m'', m']\\
\end{align*}
Let's say preference list of men is
\begin{align*}
    m &: [w , w' , w'']\\
    m'&:[w , w' , w'']\\
    m''&: [w', w, w'']
\end{align*}

If we run the Gale-Shapley on the preference lists above, we obtain the final match to be $m-w, m''-w', m'-w''$

In this case $m$ is higher in preference list of other women. But since $w$ is first preference of $m$, and $m$ proposes first in Gayle Shapely algorithm, $w$  will accept $m$'s proposal and $w'$ will accept $m''$'s proposal. The algorithm still produces stable match even though $w$ does not get her first preference. 

Now let's say $w$ lies about her preference. 

\begin{align*}
    w &: [m'' , m' , m]\\
    w'&: [m, m'', m']\\
    w''&:[m, m'', m']\\
\end{align*}

If we run the Gale-Shapley on the updated preference list of women if $w$ lies, we obtain the match to be $m-w', m''-w, m'-w''$, noticing that w1 ends up with her first preference.

Hence $w$ was matched $m''$ which she prefers higher than both $m$ and $m'$. This proves that lying about preference can increase the chances to get a better match. 
\newpage

\exercise

\end{document}