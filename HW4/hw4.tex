\documentclass{homeworg}

\title{CSEP 521, Winter 2021: Homework 4}
\author{Krishan Subudhi (ksubudhi@uw.edu) - 2040900}
\usepackage[table]{xcolor}
\usepackage{booktabs}
\usepackage{multirow}
\usepackage{pythonhighlight}
\usepackage{enumitem}
\usepackage{array}

\usepackage{graphicx}
\graphicspath{ {./images/} }
\usepackage{placeins}

\let\Oldsubsection\subsection
\renewcommand{\subsection}{\FloatBarrier\Oldsubsection}

\begin{document}

\maketitle

\exercise
\emph{Q: Show that the number 1048579 = 220 + 3 is not prime without factoring the number}

Fermat's theorm : 
\[a^{p-1} \mod p = 1,  \text{if $p$ is prime.}\]

$p = 1048579 = 2^{20} + 3$

Let's take $a = 2$. Now if we can prove $2^{2^{20}+2} \mod 1048579$ is not 1, then p is not a prime.

\begin{align*}
2^{2^3} \mod 1048579 &= (2^{2^2} \mod 1048579 \ast 2^{2^2} \mod 1048579) \mod 1048579 &= 256\\
2^{2^4} \mod 1048579 &= (2^{2^3} \mod 1048579 \ast 2^{2^3} \mod 1048579) \mod 1048579 &= 65536\\
2^{2^5} \mod 1048579 &= (2^{2^4} \mod 1048579 \ast 2^{2^4} \mod 1048579) \mod 1048579 &= 1036291\\
2^{2^6} \mod 1048579 &= (2^{2^5} \mod 1048579 \ast 2^{2^5} \mod 1048579) \mod 1048579 &= 1048147\\
2^{2^7} \mod 1048579 &= (2^{2^6} \mod 1048579 \ast 2^{2^6} \mod 1048579) \mod 1048579 &= 186624\\
\texttt{...}\\
\texttt{...}\\
2^{2^{19}} \mod 1048579 &= (2^{2^{18}} \mod 1048579 \ast 2^{2^{18}} \mod 1048579) \mod 1048579 &= 870510\\
2^{2^{20}} \mod 1048579 &= (2^{2^{19}} \mod 1048579 \ast 2^{2^{19}} \mod 1048579) \mod 1048579 &= 588380
\end{align*}

Hence ,

\[
2^{2^{20}+2} \mod 1048579 = ( {2^{2^{20}} \mod 1048579 \ast 2^{2} \mod 1048579} ) \mod 1048579 = 256362
\]

Since $256362 \ne 1$, the number 1048579 is not a prime.

\newpage
\exercise

A hash table of size m is used to store n items, with $n \le m/2$, so the load factor is at most $\frac{1}{2}$

Open addressing is used for collision resolution.

\emph{a) Assuming uniform hashing, show that for $i =1, 2,. .., n$, the probability that the $i$-th insertion requires strictly more than k probes is at most $2^{-k}$}

P ($i$-th insertion requires strictly more than k probes is at most $2^{-k}$)

= P ( first $k$ hashes in $i$-th iteration end up in collision)
\begin{align*}
P &=\left(\frac{i-1}{m}\right)^k\\
&\le \left(\frac{n}{m}\right)^k\\
&\le \left(\frac{m/2}{m}\right)^k\\
&\le 2^{-k}
\end{align*}

\emph{b) Show that for $i =1, 2,. .., n$, the probability that the $i$-th insertion requires more than $2 log n$ probes is at most $1/n^2$.}
We previously showed in (a) that ,

$$P (i-\text{th insertion requires strictly more than k probes}) \le 2^{-k}$$
Hence for $k = 2 log n$,
\begin{align*}
    P (i-\text{th insertion requires strictly more than $2 log n $ probes})&\le 2^{-2 logn}\\
    &\le \frac{1}{n^2} 
\end{align*}

Hence, the probability that the $i$-th insertion requires more than $2 log n$ probes is at most $1/n^2$


\textbf{c,d Definitions}:

$X_i$ = the number of probes required by the $i$-th insertion
As per (b), $P({X_i > 2 log n}) \le 1/n^2$ 

$X = max_{1\le i\le n} X_i$ = maximum number of probes required by any of the $n$ insertions

\emph{c) Show that $Pr({X> 2 log n}) \le 1/n$}

\begin{align*}
    P({X> 2 log n}) &= P((X_1> 2 log n) \cup (X_2> 2 log n) \cup... \cup (X_n> 2 log n)) \\
    &\le P((X_1> 2 log n) + (X_2> 2 log n) +... + (X_n> 2 log n)) \\
    &\le n \ast  \frac{1}{n^2} \\
    &\le \frac{1}{n} 
\end{align*}
Proved
\end{document}